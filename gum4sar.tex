% main.tex - University of Warwick Reports / Dissertations / Projects and 
% 
% 
% Author - Chris Quinn 28/06/2020
% 
%
% A template for students and masters dissertations, flexible for your
% needs.
%
% This is the main .tex which will tell the compiler to include everything, 
% each chapter/section is then in folders for convenience, as you include more 
% images it can get harder and harder to manage.
%
% First things first, declaration of the document class along with the packages % we need.
%
% P.S. Sorry about the back to the future references, I used them to show some 
% example work possible

% Original template by Chris Quinn, University of Warwich
% Adapted to FOE ASU template by Mohamed A. Rashad


\documentclass[pdftex,12pt,a4paper,oneside]{article}
%Can change the pt, papersize etc.

\usepackage{amsmath} %For both in-line and equation mode
\usepackage{amssymb}
\numberwithin{equation}{section} %Numbering of our equations per section
\usepackage{algorithm}
\usepackage{algorithmic} %Algorithm styles, need to be nested for the example shown
\usepackage{fancyhdr} %For our headers
\usepackage{graphicx} %Inserting images
\usepackage{subcaption}
\usepackage{lipsum}  %Blank text fill, delete me when finished
\usepackage{setspace} %Spacing on the front page for crest and titles
\usepackage[]{fncychap} % Styles can be Sonny, Lenny, Glenn, Conny, Rejne, Bjarne and Bjornstrup
\usepackage[hyphens]{url} %Deals with hyphens in urls to make them clickable
\usepackage{xcolor} %Great if you want coloured text
\usepackage{tabularx}
\usepackage{appendix} %Take a wild guess slick
\usepackage{pdflscape}
\usepackage[nodayofweek]{datetime}

\definecolor{BUR}{rgb}{0.8, 0.0, 0.0}
\definecolor{blue(pigment)}{rgb}{0.2, 0.2, 0.6}
\definecolor{green1}{RGB}{221,242,151}
\definecolor{ao}{rgb}{0.0, 0.5, 0.0}

%KEEP THIS ONE LAST it's quite buggy, it allows you to click on links within the pdf and web links without changing the colour. The mouse cursor simply changes its icon to indicate to the user. Great tool - still awkward
\usepackage[hidelinks]{hyperref}
\usepackage[absolute,overlay]{textpos}
%\usepackage{subfig}



%This will tell the compiler to do the header style, page and spacing between the header and text
\fancyhf{}
\pagestyle{fancy}
\fancyfoot[C]{\thepage}
\renewcommand{\headrulewidth}{0.2pt}


%%%%%%%%%%%%%%%%%%%%%%%%%% DOCUMENT STARTS %%%%%%%%%%%%%%%%%%%%%%%%%%%%%



%Lets begin the document, some chapters have examples in to give you an idea 
\begin{document}

% !TEX root =  ../Report.tex

\thispagestyle{empty}

\begin{spacing}{1.3}
	\begin{textblock*}{10.cm}(.2cm, .5cm)
		\includegraphics[width = .8\linewidth]{./CANVAS_logo.png}
	\end{textblock*}
	\begin{textblock*}{8.5cm}(10.5cm, .7cm)
		\noindent
		ESA contract no.: 4000147647/24/I-KE \\
		Document no.: ESA-CANVAS-TN3-GUM-2025/10 \\
		Issue/Rev nr.: 0.2 \\
		Date: \today \\
	\end{textblock*}
	\begin{center}
		\includegraphics[width = .7\linewidth]{./ESA_logo.png}
	\end{center}
	\vspace{16mm}
	\begin{center}
		\textbf{\begin{LARGE}
                  GUM principles applied to SAR Calibration and Validation
		\end{LARGE}}
		\vspace{3mm}
	\end{center}
	\begin{center}
          {\large Authors: M. W. Hansen$^{1}$, M. Portabella$^{2}$, G. Grieco$^{3}$,
          F. Dinessen$^{3}$, Y. Larsen$^{4}$, E. Malnes$^{4}$, A. S. Rabaneda$^{1}$ \\
		$^{1}$ Norwegian Meteorological Institute (MetNO) \\
		$^{2}$ Barcelona Expert Center (BEC ICM-CSIC)\\
		$^{3}$ Institute of Marine Sciences (ISMAR-CNR)\\
		$^{4}$ Norwegian Research Centre (NORCE)}
		\vspace{8mm}
	\end{center}
	\begin{center}
	     {\LARGE \textbf{Calibration and Validation for SAR (CANVAS)}}
	\end{center}
\end{spacing}

\pagenumbering{arabic}



\tableofcontents
\vspace{2cm}

\begin{center}
\line(1,0){350}
\end{center}

\newpage

\include{./abstract}

\section{The GUM Principles - Introductory Review}

\section{Application of the GUM Principles to SAR Remote Sensing}

\begin{itemize}
  \item Generate documentation about the application of GUM principles to SAR.
\end{itemize}

\lipsum[1]

% \section{Applicability of Metrology Standards to SAR Cal/Val procedures}

\subsection{Uncertainty in Radiometric Calibration}

\lipsum[2-3]

\subsection{Uncertainty in Geometric Calibration}

\subsection{Uncertainty in Polarimetric Calibration}

\subsection{Uncertainty in Interferometric Calibration}

\subsection{Uncertainty in Doppler Shift}

\subsubsection{Measurand and measurement model}

The geophysical Doppler shift is the measurand, defined as \dots.

The geophysical Doppler frequency shift is calculated as
\begin{equation}
  \varpi_g = \varpi_{\rm dc} - \varpi_{\rm geo} - \varpi_{\rm e} + \Delta\varpi^*,
  \label{eq:DCcomponents}
\end{equation}

\noindent where $\varpi_{\rm dc}$ is the Doppler centroid frequency shift,
$\varpi_{\rm geo}$ is a geometric term related to the antenna pointing via the
satellite's orbit and attitude, $\varpi_{\rm e}$ is an electronic term
resulting from antenna mis-pointing, and $\Delta\varpi^*$ is the residual error.
The estimate of each term comes with uncertainty as described in the following
sub-sections.

\subsubsection{Traceability Diagram}

\subsubsection{Uncertainty of the Doppler Centroid Frequency Shift}

The Doppler centroid frequency shift (DC) is usually estimated from the azimuth
power spectrum of the SAR beam. For ASAR data published at
\url{https://data.met.no/dataset/e19b9c36-a9dc-4e13-8827-c998b9045b54}, the
estimation algorithm is based on \cite{bamler91}, but extended to correct the
azimuth spectra for energy aliased from neighboring areas \cite{hansen26}. The
same model is used to calculate the standard deviation of the DC estimate,
which gives the standard uncertainty, $\sigma_{\rm dc}$, of $\varpi_{\rm dc}$.
Figure~\ref{fig:dc} shows $\sigma_{\rm dc}$ and $\varpi_{\rm dc}$ of an Envisat
ASAR acquisition over the Amazon rain forest on
2010-02-19T14:02:27.240842+00:00.

\begin{figure}
  \centering
  \includegraphics[width=0.3\paperwidth]{figures/dc}
  \includegraphics[width=0.3\paperwidth]{figures/dc_std}

  \caption{Doppler centroid frequency shift (DC) estimates (Hz; left) and their
  standard uncertainty (Hz; right), given by the experimental standard deviation.}

  \label{fig:dc}
\end{figure}

\subsubsection{Uncertainty of the Estimates of Doppler Shifts caused by Mis-Pointing}

\begin{figure}
  \centering
  \includegraphics[width=0.3\paperwidth]{figures/range_bias_LUT_2012-01-27_2012-02-16_VV.dcgeo_residual}
  \includegraphics[width=0.3\paperwidth]{figures/range_bias_LUT_2012-01-27_2012-02-16_VV.oacorr_residual}
  \includegraphics[width=0.3\paperwidth]{figures/orbital_residual_2012-01-27_2012-02-16_VV}
  \includegraphics[width=0.3\paperwidth]{figures/antenna_mispointing_2012-01-27_2012-02-16_VV}

  \caption{Distributions of global Doppler centroid anomaly (DCA; VV
  polarization) measurements over land between the 27th January and 16th
  February 2012 following (a) subtraction of the Doppler shift resulting from a-priori
  orbit and attitude estimates given in auxiliary orbit and attitude data files
  from DC, and (b) after attitude correction by manual fit. Panel (c) shows the
  orbital variation of the residual geometric Doppler frequency shift,
  $\varpi^{\rm r}_1(\gamma_h)$ (dots), and the interpolated signal,
  $\varpi^{\rm r}_2(\gamma_h)$ (solid line), and panel (d) shows the DCA
  distribution attributed to electronic mis-pointing, $\varpi_e$, and random
  noise, $\Delta\varpi$, after also correcting for the orbital variation.}

  \label{fig:mispointing-corr}
\end{figure}

Figure~\ref{fig:mispointing-corr} shows the DC measurements over land after
step-wise subtraction of (a) the Doppler shift resulting from a-priori orbit
and attitude estimates given in auxiliary orbit and attitude data files and (b)
after attitude correction obtained by manual fit.
Figure~\ref{fig:mispointing-corr}(c) shows the remaining residual as a function
of the satellite hour angle, i.e., its orbit, and
Figure~\ref{fig:mispointing-corr}(d) shows the DC measurements, or Doppler
Centroid Anomaly (DCA), after subtraction of the orbital bias. The uncertainty
of the corrections can be described by the variance around the mean DCA in bins
of the sensor view angle, as shown in
Figure~\ref{fig:antenna-pattern-with-errorbars} for the third sub-swath, where
the error bars represent the experimental standard deviation in each bin.

\begin{figure}
  \centering
  \includegraphics[width=0.6\paperwidth]{figures/antenna-pattern-with-errorbars}

  \caption{Antenna pattern DCA bias with error bars for the third sub-swath of
  the data shown in Figure~\ref{fig:mispointing-corr}d.}

  \label{fig:antenna-pattern-with-errorbars}
\end{figure}

\noindent {\bf Uncertainty of the Antenna Pattern.} Equation E.3 from GUM?

$z=f(w_1, w_2, ..., w_N)$

$\displaystyle\sigma_z^2 = \sum_{i=1}^N (\frac{\partial f}{\partial w_i})^2\sigma_i^2 + 2\sum_{i=1}^{N-1}\sum_{j=i+1}^N \frac{\partial f}{\partial w_i}\frac{\partial f}{\partial w_j}\sigma_i\sigma_j\rho_{ij}$

What about the Mean vs the Shape?

\subsubsection{Uncertainty of the Geophysical Doppler Shift}

\begin{equation}
  \Delta\varpi^* = \Delta\varpi_{\rm lb} + \Delta\varpi,
  \label{eq:final-residual}
\end{equation}

{\bf By propagation of errors.}

{\bf By considering the variance over land.}

See Figure~\ref{fig:fg_plot}. The mean is here $-0.10$~Hz, with standard deviation $6.9$~Hz.

\begin{figure}
  \centering
  \includegraphics[width=0.6\paperwidth]{figures/fg_plot}

  \caption{Remaining bias after subtraction of $\varpi_{\rm geo}$ and $\varpi_{\rm e}$.}

  \label{fig:fg-plot}
\end{figure}

\subsection{Uncertainty in Auxiliary Data}

\subsubsection{E.g., Uncertainty in Wave Spectrum used to Estimate Azimuth Cut-Off}

\subsubsection{Uncertainty in Auxiliary Wind Directions used for SAR Wind Retrieval}

\section{Error Propagation}

\subsection{General Approach}
\subsection{Example: SAR Wind Retrieval}

Demonstrate identification, modeling and propagation of uncertainties.

%\subsection{Example: Wet Snow Retrieval}
%
%Demonstrate identification, modeling and propagation of uncertainties.
%
%\subsection{Example: Soil Moisture}
%
%Demonstrate identification, modeling and propagation of uncertainties.
%
%\subsection{Example: Sea State}
%
%Demonstrate identification, modeling and propagation of uncertainties.


\bibliographystyle{ieeetr}
\bibliography{bibliography}

\end{document}
